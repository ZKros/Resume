\begin{center}
	\textbf{\Huge Daniel da Silva Pinto Pereira} \\ \vspace{5pt}
	\small \faPhone* \texttt{+55 (27)996915861} \hspace{1pt} $•$
	\hspace{1pt} \faEnvelope \hspace{2pt} \texttt{danielsilva25@live.com} \hspace{1pt}
	% \hspace{1pt} \faYoutube \hspace{2pt} \texttt{harshibar} \hspace{1pt} $|$
	% \hspace{1pt} \faMapMarker* \hspace{2pt}\texttt{Vila Velha, ES, Brasil}
	\\ \hspace{1pt} \faLinkedin \hspace{2pt} \texttt{https://www.linkedin.com/in/daniel-pereira-4960621b3/} $•$
	\hspace{1pt} \faGithub \hspace{2pt} \texttt{https://github.com/ZKros}
	\\ \vspace{-3pt}
\end{center}

%-----------RESUME-----------
\section{RESUMO}
\resumeSubHeadingListStart

\resumeItem{Desenvolvedor Fullstack Junior com experiência em Angular, Java, Kotlin e integração de APIs REST. Atuação em desenvolvimento de interfaces responsivas, otimização de aplicações, testes unitários e versionamento com Git. Certificado em Clean Code, SOLID e PostgreSQL. Perfil colaborativo, focado em boas práticas de programação e entrega de soluções eficientes.}
\resumeSubHeadingListEnd

%-----------EXPERIENCE-----------
\section{EXPERIÊNCIA PROFISSIONAL}
\resumeSubHeadingListStart

\resumeSubheading
{Quality Automação}{Jun. 2022 -- Fev. 2025}
{Desenvolvedor Fullstack Junior}{}
\resumeItemListStart
\resumeItem{Desenvolvi interfaces responsivas em Angular e SCSS, garantindo adaptação a múltiplos dispositivos.}
\resumeItem{Criei componentes reutilizáveis e integrei APIs RESTful em tempo real, otimizando a comunicação entre sistemas.}
\resumeItem{Apliquei técnicas de otimização como lazy loading e compressão, reduzindo requisições HTTP e melhorando performance.}
\resumeItem{Utilizei RxJS e NgRx para gerenciamento eficiente de estado em aplicações Angular.}
\resumeItem{Implementei testes unitários com Jasmine e Karma, elevando a confiabilidade do código.}
\resumeItem{Desenvolvi backend orientado a objetos com Java e Kotlin.}
\resumeItem{Gerei relatórios exportáveis (CSV, Excel, PDF) com filtros personalizados para análise de dados.}
\resumeItem{Utilizei Git e Jira para controle de versão e gestão de tarefas.}
\resumeItemListEnd
\resumeSubHeadingListEnd

\resumeSubHeadingListStart
\resumeSubheading
{Tracomal}{Nov. 2021 -- Jun. 2022}
{Estagiário de TI Infra}{}
\resumeItemListStart
\resumeItem{Auxiliei no atendimento a usuários e na instalação de softwares.}
\resumeItem{Colaborei na manutenção e configuração de firewall, banco de dados e servidores.}
\resumeItemListEnd
\resumeSubHeadingListEnd

\resumeSubHeadingListStart
\resumeSubheading
{Prysmian Group}{Set. 2020 -- Ago. 2021}
{Estagiário de TI Infra}{}
\resumeItemListStart
\resumeItem{Prestei suporte a usuários em hardware, Windows 10, SAP e coleta de dados.}
\resumeItem{Gerenciei dados e armazenamento, além de suporte a redes e plataformas de telecomunicações.}
\resumeItemListEnd
\resumeSubHeadingListEnd

%-----------EDUCATION-----------
\section {FORMAÇÃO ACADÊMICA}
\resumeSubHeadingListStart

\resumeSubheading
{Universidade Centro Leste - UCL}{Jul. 2024}
{Tecnólogo em Analíse e Desenvolvimento de Sistemas}
{Serra, ES}
\resumeSubHeadingListEnd

%-----------COURSE-----------
\section {CURSOS E CERTIFICAÇÕES}

\resumeSubHeadingListStart
\vspace{-1pt}\item
\begin{tabular*}{\textwidth}[t]{l@{\extracolsep{\fill}}r}
	\textbf{Udemy} & {\color{dark-grey}\small}\vspace{1pt}\\ % top row of resume entry
	\textit{Certificado Clean Code na Prática (Código Limpo)} & {\color{dark-grey} \small Maio. 2023}\\ % second row of resume entry
	\textit{Certificado SOLID - Os 5 Princípios Para as Boas Práticas da POO} & {\color{dark-grey} \small Mar. 2024}\\ % second row of resume entry
	\textit{Certificado Curso completo de PostgreSQL! Do Básico ao Avançado} & {\color{dark-grey} \small Jul. 2022}\\ % second row of resume entry
\end{tabular*}\vspace{-4pt}
\resumeSubHeadingListEnd

%-------PROJECTS AND REALIZATIONS--------
%\section{PROJETOS E REALIZAÇÕES}
%\resumeSubHeadingListStart
%\resumeSubheading
%{Iniciação Científica Voluntária - Universidade Vila Velha}{Jul. 2022 -- Jul. 2023}
%{NER em Acórdãos da TNU: Pedidos de Uniformização}{}

%\resumeSubheading
%{ICPC - International Collegiate Programming Contest}{Nov. 2020}
%{XXVI Maratona de Programação}{}
          
%\resumeSubHeadingListEnd

%-----------PROGRAMMING SKILLS-----------
\section{HABILIDADES TÉCNICAS}
\begin{itemize}[leftmargin=0in, label={}]
	\small{\item{
		\textbf{Linguagens de Programação:}
		{HTML, CSS, JavaScript, TypeScript, Java e Kotlin.}
		\vspace{2pt} \\
				
		\textbf{Frameworks:}
		{Angular, Spring Boot e Svelte.}
		\vspace{2pt} \\
						     
		\textbf{Banco de Dado:}
		{PostgreSQL.}
		\vspace{2pt} \\

		\textbf{Ferramentas:}
		{Jira, Git, GitHub, Postman e Insomnia.}
		\vspace{2pt} \\

		\textbf{Testes:}
		{Jasmine, Karma, Testes Unitários e de Integração.}
		\vspace{2pt} \\
						     
		\textbf{Idiomas:}
		{Português Nativo e Inglês Intermediário.}
	}}
\end{itemize}